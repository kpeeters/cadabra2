\cdbproperty{LaTeXForm}{\sf valid \LaTeX{} expression}

Changes the way in which symbols are displayed in the graphical
interface. Example:
\begin{screen}{1,2}
\del{#}::LaTeXForm("\partial").
\del_{m}(A);
\end{screen}
This prints $\partial_{m}(A);$ in the notebook, despite the fact
that \verb|\del| is not a \LaTeX{} command.

If you use this property to make a symbol printable, make sure to
declare it \emph{before} any other properties are declared, otherwise
the notebook will not know how to display the symbol and produce an
error message.

Note that the property is attached to a pattern (\verb|\del{#}| in
this case) which matches the expression in which the replacement has
to be made. If the pattern matches, the replacement will be done on
the head symbol (\verb|\del| in this case). A pattern \verb|\del|
without the argument wildcard \verb|#| would only replace
when \verb|\del| occurs without any arguments (as in 
e.g.~\verb|\del + A|).

These settings have no effect in the command-line version.

