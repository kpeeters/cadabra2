\cdbalgorithm{asym}{}

Anti-symmetrise a product or tensor in the indicated objects. This works
both with normal objects, as in
\begin{screen}{1,2}
A B C;
@asym!(%){A,B,C};
1/6 A B C - 1/6 A C B - 1/6 B A C 
          + 1/6 B C A + 1/6 C A B - 1/6 C B A;
\end{screen}
as well as with indices. When used with indices, remember to also
indicate whether you want to symmetrise upper or lower indices, as in
the example below.
\begin{screen}{1,2}
A_{m n} B_{p};
@asym!(%){ _{m}, _{n}, _{p} };
1/6 A_{m n} B_{p} - 1/6 A_{m p} B_{n} - 1/6 A_{n m} B_{p}
     + 1/6 A_{n p} B_{m} + 1/6 A_{p m} B_{n} - 1/6 A_{p n} B_{m};
\end{screen}
Symmetrisation (i.e.~using plus signs for all terms) is handled by
the \subscommand{sym} algorithm.  ~

\cdbseealgo{sym}
\cdbseealgo{young_project}
\cdbseealgo{young_project_tensor}
